%_______________________________________________________________________________
%class
%_______________________________________________________________________________
%\documentclass[a4paper,11pt,onecolumn,final,german,openbib]{scrbook}
\documentclass[a4paper,11pt,oneside,final,german,openbib,pdftex]{scrbook}
%_______________________________________________________________________________
% page borders
%_______________________________________________________________________________
\addtolength{\headheight}{2cm}
%\addtolength{\topmargin}{2cm}
\setlength{\oddsidemargin}{1.0cm}
\setlength{\evensidemargin}{0.5cm}
\setlength{\textwidth}{14.3cm}
\setlength{\parindent}{0mm}

%_______________________________________________________________________________
% packages
%_______________________________________________________________________________
\usepackage{german}
\usepackage{amsmath, amssymb}
\usepackage[utf8]{inputenc}
\usepackage{graphicx}
\usepackage{enumerate}
\usepackage{multirow}
\usepackage{subfigure}
\usepackage{dsfont}
\usepackage{slashed}
\usepackage{textcomp}
\usepackage{url}
\usepackage{hyperref}
\usepackage{endnotes}


%_______________________________________________________________________________
% bold fonts for headings
%_______________________________________________________________________________
\font\afont=cmssbx10 scaled \magstep5     % for the title
\font\bfont=cmssbx10 scaled \magstep4     % for chapter headings
\font\cfont=cmssbx10 scaled \magstep3
\font\dfont=cmssbx10 scaled \magstep2     % for section headings and author name
\font\efont=cmssbx10 scaled \magstephalf

%_______________________________________________________________________________
% index depth
%_______________________________________________________________________________
\setcounter{secnumdepth}{3}
\setcounter{tocdepth}{3}

%_______________________________________________________________________________
% new commands
%_______________________________________________________________________________
\newcommand{\demi}{\frac{1}{2}}

%_______________________________________________________________________________
% renewed commands
%_______________________________________________________________________________
% \renewcommand{\topfraction}{1.}       % this is important for figure placement
% \renewcommand{\bottomfraction}{1.}
\makeatletter
\renewcommand\paragraph{\@startsection{paragraph}{4}{\z@}%
  {-3.25ex\@plus -1ex \@minus -.2ex}%
  {1.5ex \@plus .2ex}%
  {\normalfont\normalsize\bfseries}
}
\makeatother

%_______________________________________________________________________________
% special words, hyphenation
%_______________________________________________________________________________
\hyphenation{Ba-che-lor-ar-beit}

\pagestyle{empty}
\pagestyle{headings}
%for changing the style on a specific page use \thispagestyle{e.g., empty}

%_______________________________________________________________________________
%_______________________________________________________________________________
\begin{document}
\pagenumbering{roman}

%_______________________________________________________________________________
\begin{titlepage}
  \vspace*{6mm}
  \begin{center}
     {\afont Titel der Bachelorarbeit}
     \\[3.5cm]
     {\large von}
     \\[3.5cm]
     {\dfont Martin Sobotzik}
     \\[2cm]
     {\large Bachelorarbeit in Physik \ rtm/\\
        vorgelegt dem Fachbereich Physik, Mathematik und Informatik (FB 08) \/\\
        der Johannes Gutenberg-Universit\"at Mainz \/\\
        am 1. April 2012}
   \end{center}
   \vfill
   1. Gutachter: Prof. Dr. Lebeim Elfenbeinturm\\	
   2. Gutachter: Prof. Dr. Habe D\"unkel \\
   \vfill
\end{titlepage}

\thispagestyle{empty}
Ich versichere, dass ich die Arbeit selbstst\"andig verfasst und keine 
anderen als die angegebenen Quellen und Hilfsmittel benutzt sowie 
Zitate kenntlich gemacht habe.
\\
\\[3.5cm] 
Mainz, den [Datum] [Unterschrift]
\vfill
\noindent 
Johanna Musterfrau\\
KOMET\\
Institut f\"ur Physik\\
Staudingerweg 7\\
Johannes Gutenberg-Universit\"at
D-55099 Mainz\\
{\tt msobotzi@students.uni-mainz.de}

%_______________________________________________________________________________
\renewcommand\contentsname{Inhaltsverzeichnis}
\renewcommand\figurename{Abbildung}
\renewcommand\tablename{Tabelle}
\tableofcontents
\clearpage

\mainmatter
\sloppy

%_______________________________________________________________________________
\chapter{Einleitung}

{\em Dieses Dokument richtet sich an Studierende am Fachbereich 08 im 
Studiengang Bachelor of Science (Physik). Sie finden hier Beispiele 
f\"ur eine m\"ogliche Gliederung Ihrer Arbeit und Hinweise zur 
Strukturierung des Inhalts. Selbstverst\"andlich sollen Sie diese 
Gliederung nach den Gegebenheiten Ihrer Bachelorarbeit anpassen. 
Besprechen Sie rechtzeitig mit Ihrem Betreuer, ob Ihr Entwurf sinnvoll 
ist. Holen Sie sich auch Anregungen zur Gestaltung von Abschlussarbeiten 
aus der Literatur (siehe z.\ B.\ \cite{EbelBliefert}).
\medskip

Sofern Sie sich dazu entscheiden, Ihr Dokument in \LaTeX\ zu erstellen, 
k\"onnen Sie diese Datei als Vorlage verwenden. Fast die gesamte 
Literatur in der Physik verwendet \LaTeX, vor allem wegen der 
ausgezeichneten M\"oglichkeiten f\"ur das Formelschreiben.
}
\bigskip

In der Einleitung Ihrer Bachelorarbeit sollte das Thema der Arbeit 
m\"oglichst allgemeinverst\"andlich eingef\"uhrt werden. Gehen Sie 
dabei auch auf das weitere Umfeld der Arbeit ein und erl\"autern Sie, 
warum Aufgabenstellung und Herangehensweise interessant sind. Auch 
die weitere Gliederung kann angesprochen werden, um dem Leser einen 
ersten \"Uberblick \"uber den nachfolgenden Text zu geben.

%_______________________________________________________________________________
\chapter{Experimenteller Aufbau am MAMI}




%_______________________________________________________________________________

Der Mainzer Mikrotron (MAMI) war zur Zeit meiner Bachelorarbeit ein mehrstufiger Rennbahn-Teilchenbeschleuniger (RTM\footnote{Race-Track-Microtron}) für Elektronenstrahlen und stand verschiedenen Arbeirsgruppen
 für Experimente zur Verfügung. Die Anlage befand sich auf dem Gelände des Instituts für Kernphysik (KPh) der Johannes Gutenberg-Universität 
und bestand aus mehreren Hallen. 
\newline
\newline

Die A2-Kollaborations untersuchte unter anderem die Struktur von Nukleonen mittels reeller Photonen. Diese Photonen werden durch Bremsstrahlung des MAMI-Elektronenstrahls erzeugt und ihre Energie wird durch eine Photonenmarkierungsanlage (Tagger\footnote{to tag: markieren}) bestimmt.


\section{Der MAMI-Beschleuniger}
1979 wurde das MAMI erstmals in Betrieb genommen und bestand damals nur aus einem einzelnen RTM, womit eine maximale Elektronenenergie von 14 MeV erreicht werden konnte. 
Im Laufe der Jahre wurde das MAMI um zwei weitere RTMs und einem HDSM \footnote{Harmonic Double Sided Microtron} erweitert, wodurch eine Elektronenenergie von 1,5 GeV erreicht werden konnte.\cite{KPh11G} \newline

\subsection{Funktionsprinzip des MAMI}
Um unpolarisierte Elektronen zu erzeugen, wurde eine Glühkathode auf 1000°C erhitzt. Dadurch konnten Elektronen den Heizdraht, aufgrund ihrer thermischen Bewegung, verlassen. Diese Elektronen wurden durch ein elektrisches Feld, welches durch die heiße Kathode und einer Anode, erzeugt wurde, zur Anode beschleunigt und traten dann durch ein Loch in der Anode aus und wurden weiter durch einen Linearbeschleuniger mit einer Frequenz von 2,45 GHz auf ca. 3,5 MeV beschleunigt. \cite{Un08} Diese Frequenz ist für das MAMI typisch und machte es zu einem Dauerstrich-Elektronen-Beschleuniger. Das heißt die Frequenz, mit der die Elektronen-Pakete auftraten, war größer, als die Frequenz, mit der die Detektoren einzelne Events auflösen konnten und somit wirkte der Strahl für die Detektoren kontinuierlich.
\newline
Da die Elektonen mit einem Linearbeschleuniger nur einige MeV pro Meter beschleunigt werden k\"onnen, und man keine kilometerlangen Strecke bauen wollte, entschied man sich daf\"ur, die Elektronen mehrmals durch den gleichen Beschleunigerabschnitt zu beschleunigen. Dazu wurden sie nachdem sie beschleunigt wurden, durch zwei 180° Dipole so umgeleitet, dass sie wieder am Anfang des Beschleunigerabschnitts waren und diese Bahn abermals durchlaufen konnten. Eine phasengerichtete R\"uckkopplung ist allerdings nur m\"oglich, wenn die statische und die dynamische Koh\"arenzbedingung erf\"ullt sind. Damit die statische Koh\"arenzbedingung erf\"ullt ist, muss die L\"ange der ersten vollst\"andigen Bahn ein ganzzahliges Vielaches der beschleunigten Hochfrequenz sein. F\"ur die dynamische Koh\"arenz muss die L\"angendifferenz von zwei aufeinander folgenden Uml\"aufen ebenfalls ein ganzzahliges Vielfaches de Wellenl\"ange sein\cite{Un08}. Diese Bedingungen gaben ebenfalls die Grenzen f\"ur den maximal m\"oglichen Energiegewinn jeder Stufe an. 
\newline
\newline
Wie bereits erw\"ahnt besitzt MAMI drei dieser RTMs. Die erste Stufe MAMI A bestand aus zwei RTMs mit 18 bzw. 51 Uml\"aufen. Die zweite Stufe MAMI B bestand aus dem, zu diesem Zeitpunkt, gr\"o{\ss}ten RTM der Welt mit 90 Uml\"aufen und Dipolen mit einer Breite von jeweils 5 m, wodurch sie 450 t schwer waren. Damit waren auch die technischen Grenzen erreicht.\cite{KPh11F}
\newline
Um nun aber trotzdem h\"ohere Energien zu erreichen, musste sich ein neues Konzept \"ubelegt werden. MAMI C war folglich kein RTM mehr, sondern ein HDSM. Das hei{\ss}t, es bestand aus vier 90° Dipolen, welche jeweils 250 t schwer waren und einem zus\"atzlichen Linearbeschleuniger. F\" dieses HDSM wurde der erste Linearbeschleuniger der Welt gebaut, der mit einer Fequenz von 4,9 GHz betrieben werden konnte betrieben wurde er allerdings, wie die beiden voherigen RTMs mit einer Frequenz von 2,45 GHz/.
\newline
 Der Elektronenstrahl hatte am Ende der Beschleunigung eine Energie von ca. 1,5 GeV, diese konnte in etwa 15 MeV Schriten eingestellt werden, und einen Durchmesser im Mikrometerbereich, was sehr gute Voraussetzungen f\"ur Pr\"azisionsexperimente sind.\cite{KPh07}. 




\section{Versuchsaufbau}

Wenn Sie an einem experimentellen Thema arbeiten, beschreiben Sie 
den Versuchsaufbau, auch wenn Sie an einem bereits vorhandenen 
Versuch arbeiten, soweit dies f\"ur Ihre spezielle Fragestellung 
relevant ist. 

\section{Methoden}

Entsprechend kann es bei einer theoretischen Arbeit sinnvoll sein,
die L\"osungsmethoden in einem eigenen Kapitel zu beschreiben.

\section{Ergebnisse}

Hauptteil Ihrer Arbeit ist das Kapitel (oder die Kapitel) mit den 
Ergebnissen. Bei einer theoretischen Arbeit kann damit auch 
die Herleitung von Formeln oder die Beschreibung eines Computerprogramms 
gemeint sein.

%_______________________________________________________________________________
\chapter{Zusammenfassung und Ausblick}

In der Zusammenfassung sollten Sie in knapper Form die Aufgabenstellung 
und die wichtigsten Ergebnisse rekapitulieren. Es ist f\"ur die 
Gutachter hilfreich, wenn Sie ausdr\"ucklich beschreiben, worin 
Ihre eigenen Beitr\"age liegen. Scheuen Sie sich auch nicht davor 
auszusprechen, welche Untersuchungen durch die Zeitbegrenzung der 
Bachelorarbeit nicht m\"oglich waren und nutzen Sie dies als 
\"Uberleitung zu einem Ausblick auf m\"ogliche weitergehende 
Arbeiten an der Aufgabenstellung.

%_______________________________________________________________________________
\begin{appendix}
\chapter{Anhang}

\section{Tabellen und Abbildungen}

In der Regel sind die in Tabellen und Abbildungen enthalten Informationen 
so wichtig, dass sie im Hauptteil der Arbeit erscheinen sollten. Unter 
Umst\"anden sind aber erg\"anzende Tabellen und Abbildungen gut in einem 
Anhang aufgehoben. Wie im Hauptteil sollten Sie auch hier darauf achten, 
dass die in Tabellen und Figuren (siehe Abb.\ \ref{Abb:1}) dargestellte 
Information im Text angesprochen wird und selbsterkl\"arende Legenden 
vorhanden sind.
\medskip

\begin{figure}[h]
\begin{center}
%\includegraphics[scale=0.8]{BA-Abbildung1.pdf}
\end{center}
\caption{\label{Abb:1}
Feynmandiagramm f\"ur eine typische Einschleifen-Korrektur zur 
Produktion von sieben Jets in der $e^+e^-$-Vernichtung (entnommen 
aus \cite{thepnews}, mit Zustimmung der Autoren).
} 
\end{figure}


%_______________________________________________________________________________
\section{Weiterf\"uhrende Details zur Arbeit}

Manch wichtiger Teil Ihrer tats\"achlichen Arbeit ist zu technisch 
und w\"urde den Hauptteil des Textes un\"ubersichtlich machen, 
beispielsweise wenn es um die Details des Versuchsaufbaus in einer 
experimentellen Arbeit oder um den f\"ur eine numerische Auswertung 
verwendeten Algorithmus geht. Dennoch ist es sinnvoll, entsprechende 
Beschreibungen in einem Anhang Ihrer Bachelorarbeit aufzunehmen. 
Insbesondere f\"ur zuk\"unftige Arbeiten, die an Ihre Bachelorarbeit 
anschlie{\ss}en, sind dies manchmal hilfreiche Informationen.

%_______________________________________________________________________________
\chapter{Literaturverzeichnis}

Machen Sie genaue Angaben, so dass die verwendeten Literaturstellen 
eindeutig identifiziert und aufgefunden werden k\"onnen.
Bei Lehrb\"uchern \cite{Weinberg:1995mt} ist es sinnvoll, 
den Titel anzugeben, eventuell auch die Ausgabe. Bei Artikeln in 
Fachzeitschriften \cite{Moch:2001zr} ist es \"ublich, nur die 
gebr\"auchlichen Abk\"urzungen f\"ur den Titel der Zeitschrift, Band, 
Erscheinungsjahr und Seite anzugeben. Unter Umst\"anden kann es auch 
sinnvoll sein, im Internet aufgefundene Informationsquellen anzugeben, 
zum Beispiel f\"ur Software \cite{LoopTools} oder zu den Details von 
Ergebnissen gro{\ss}er experimenteller Kollaborationen. Es ist 
selbstverst\"andlich, dass Sie auch Bachelor- \cite{BA:Freund}, 
Diplom- oder Doktorarbeiten angeben, wenn Sie diese in Ihrer eigenen 
Arbeit verwendet haben.
\medskip

Im folgenden Beispiel werden die in der Datei {\tt h-physrev3.bst} 
enthaltenen Anweisungen als Stilvorlage verwendet. Andere 
M\"oglichkeiten f\"ur die Gestaltung eines Literaturverzeichnisses 
findet man im Internet: \url{http://janeden.net/bibliographien-mit-latex}.

\renewcommand{\bibname}{\bfont Literaturverzeichnis} 
\bibliographystyle{h-physrev3}
\begin{thebibliography}{99}
\bibitem[Un08]{Un08} Dissertation von Marc Unverzagt,  2008 {\em Bestimmung des Damitz-Plot-Parameters $\alpha$ für den Zerfall $ \eta 3\pi^{0} $ mit dem Crystal Ball am MAMI}
\bibitem[KPh11G]{KPh11G} Internetseite der Kernphysik {\em Mainzer Mikrotron-Geschichte} Internetseite \url{http://www.kernphysik.uni-mainz.de/379.php} (Stand 04.03.2017)
\bibitem[KPh11F]{KPh11F} Internetseite der Kernphysik {\em Funktionsprinzip des MAMI} Internetseite \url{http://www.kernphysik.uni-mainz.de/375.php} (Stand 06.03.2017)

\bibitem[KPh04]{KPh04} Prospekt des Institut für Kernphysik Internetlink \url{https://portal.kph.uni-mainz.de/de/information/introduction/prospekt.pdf} (Stand: 04.03.2017)
\bibitem[We13]{We13} Diplomarbeit von Jennifer Wettig, 2013 {\em Aufbau und Inbetriebnahme einer neuen HV-Versorgung für den Crystal Ball Detektor am MAMI}
\bibitem[KPh07]{KPh07} Pressemitteilung der KPh \url{https://www.uni-mainz.de/presse/archiv/zope.verwaltung.uni-mainz.de/presse/mitteilung/2007/2007_10_05_phys_einweihung_mami/showArticle_dtml.html} (Stand 06.03.2017)








%\cite{thepnews}
\bibitem{EbelBliefert}
H.\ F.\ Ebel, C.\ Bliefert, 
  ``Bachelor-, Master- und Doktorarbeit: Anleitungen f\"ur den 
  naturwissenschaftlich-technischen Nachwuchs,''
  Wiley-VCH, Weinheim (2009). 
\bibitem{thepnews}
  S.~Becker, D.~G\"otz, C.~Reuschle, C.~Schwan, S.~Weinzierl,  
  \url{http://wwwthep.physik.uni-mainz.de/site/news/168/}.

%\cite{Weinberg:1995mt}
\bibitem{Weinberg:1995mt}
  S.~Weinberg,
  ``The Quantum theory of fields. Vol. 1: Foundations,''
  Cambridge, UK: Univ. Pr. (1995) 609 p.

%\cite{Moch:2001zr}
\bibitem{Moch:2001zr}
  S.~Moch, P.~Uwer, S.~Weinzierl,
  %``Nested sums, expansion of transcendental functions and multiscale 
  % multiloop integrals,''
  J.\ Math.\ Phys.\  {\bf 43 } (2002)  3363-3386.
  [hep-ph/0110083].

%\cite{LoopTools}
\bibitem{LoopTools}
  T.~Hahn, 
  ``The LoopTools Site,''
  \url{http://www.feynarts.de/looptools/}.

%\cite{BA:Freund}
\bibitem{BA:Freund}
  B.~Freund, 
  Bachelorarbeit, Johannes Gutenberg-Universit\"at Mainz, 2012.

\end{thebibliography}

%_______________________________________________________________________________
\chapter{Danksagung}

... an wen auch immer. Denken Sie an Ihre Freundinnen und Freunde, 
Familie, Lehrer, Berater und Kollegen.

\end{appendix}


\end{document}  
        
        